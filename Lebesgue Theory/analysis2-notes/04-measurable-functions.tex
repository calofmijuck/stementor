\begin{center}
    \begin{tikzpicture}[
            simple/.style={red, ultra thick},
            dots/.style={semithick, dotted},
        ]
        \node[below left=1pt] (origin) at (0,0) {\(\rm{O}\)};
        \draw [-stealth] (-1, 0) -- (10, 0) node[below] {\(x\)};
        \draw [-stealth] (0, -1) -- (0, 7) node[left] {\(y\)};

        \draw[thick,domain=0.2893:9.71, smooth, variable=\x] plot ({\x}, {-3.0625 + 3.9375*\x - 0.9375*\x^2 + \x^3/16 + 2});

        \pgfmathsetmacro{\maxn}{3}
        \pgfmathsetmacro{\limit}{24}

        \draw[simple] (0.2893, 0) -- (0.6083, 0);
        \draw[simple] (0.6083, 1) -- (1, 1);
        \draw[simple] (1, 2) -- (1.5358, 2);
        \draw[simple] (1.5358, 3) -- (5, 3);
        \draw[simple] (5, 2) -- (8.4641, 2);
        \draw[simple] (8.4641, 3) -- (9, 3);
        \draw[simple] (9, 4) -- (9.3916, 4);
        \draw[simple] (9.3916, 5) -- (9.7106, 5);

        \draw[dotted] (0.6083, 0) -- (0.6083, 1);
        \draw[dotted] (1, 1) -- (1, 2);
        \draw[dotted] (1.5358, 2) -- (1.5358, 3);
        \draw[dotted] (5, 2) -- (5, 3);
        \draw[dotted] (8.4641, 2) -- (8.4641, 3);
        \draw[dotted] (9, 3) -- (9, 4);
        \draw[dotted] (9.3916, 4) -- (9.3916, 5);

        \node at (8.5, 5) {\(f(x)\)};
        \node[red] at (9.3, 1.5) {\(s_n(x)\)};
    \end{tikzpicture}
\end{center}

\section*{Measurable Functions}

Lebesgue integral을 공부하기 전 마지막 준비입니다. Lebesgue integral은 다음과 같이 표기합니다.
\[
    \int_X f \d{\mu}
\]
표기를 보면 크게 3가지 요소가 있음을 확인할 수 있습니다. 바로 집합 \(X\), measure \(\mu\), 그리고 함수 \(f\)입니다. 집합과 measure는 다루었으니 마지막으로 함수에 관한 이야기를 조금 하면 Lebesgue integral을 정의할 수 있습니다!

이제부터 다루는 measurable function 관련 내용은 일반적인 measurable space \((X, \scr{F})\)에서 논의합니다. 여기서 \(\scr{F}\)는 당연히 \(\sigma\)-algebra on \(X\)입니다.

\defn{11.13} \note{Measurable Function} Measurable space \((X, \scr{F})\)와 함수 \(f : X \ra \overline{\R}\) 가 주어졌을 때, 모든 \(a \in \R\) 에 대하여 집합
\[
    \{x \in X : f(x) > a\}
\]
가 measurable이면 \(f\)를 \textbf{measurable function}이라 한다.\footnote{일반적으로는 `measurable set의 preimage가 measurable이 될 때'로 정의합니다.}

위 사실로부터 다음을 바로 알 수 있습니다.

\cor \(\R^p\)에서 정의된 연속함수는 Lebesgue measurable이다.

\pf 임의의 \(a \in \R\) 에 대해 \(\{x : f(x) > a\}\)가 \(\R^p\)의 열린집합이므로, \(\mf{M}(m)\)의 원소가 되어 measurable이다.

위 정의를 보고 생각하다 보면 굳이 \(f(x) > a\) 로 정의해야 했나 의문이 생깁니다. \(f(x) \geq a\), \(f(x) < a\) 를 사용할 수도 있었을 것입니다.

\thm{11.15} Measurable space \(X\) 위에서 정의된 함수 \(f\)가 주어졌을 때, 다음은 동치이다.
\begin{enumerate}
    \item 모든 \(a \in \R\) 에 대하여 \(\{x : f(x) > a\}\)는 measurable이다.
    \item 모든 \(a \in \R\) 에 대하여 \(\{x : f(x) \geq a\}\)는 measurable이다.
    \item 모든 \(a \in \R\) 에 대하여 \(\{x : f(x) < a\}\)는 measurable이다.
    \item 모든 \(a \in \R\) 에 대하여 \(\{x : f(x) \leq a\}\)는 measurable이다.
\end{enumerate}

\pf 우선 (1)을 가정하고, 다음 관계식을 이용하면
\begin{equation*}
    \begin{aligned}
        \{x : f(x) \geq a\} & = f\inv\paren{[a, \infty)}                                            \\
                            & = f\inv\paren{\bigcup_{n=1}^{\infty} \paren{a + \frac{1}{n}, \infty}} \\
                            & = \bigcup_{n=1}^{\infty} f\inv\paren{\paren{a + \frac{1}{n}, \infty}}
    \end{aligned}
\end{equation*}
measurable set의 countable union도 measurable이므로 (\(\sigma\)-algebra) (2)가 성립한다. 이제 (2)를 가정하면
\[
    \{x : f(x) < a\} = X \bs \{x : f(x) \geq a\}
\]
로부터 (3)이 성립하는 것을 알 수 있다. (3)을 가정하면 위와 마찬가지 방법으로
\[
    \begin{aligned}
        \{x : f(x) \leq a\} & = f\inv \paren{(-\infty, a]}                                           \\
                            & = f\inv\paren{\bigcup_{n=1}^{\infty} \paren{-\infty, a - \frac{1}{n}}} \\
                            & = \bigcup_{n=1}^{\infty} f\inv\paren{\paren{-\infty, a - \frac{1}{n}}}
    \end{aligned}
\]
과 같이 변형하여 (4)가 성립함을 알 수 있다. 마지막으로 (4)를 가정하면
\[
    \{x : f(x) > a\} = X \bs \{x : f(x) \leq a\}
\]
로부터 (1)이 성립함을 알 수 있다.

\dots

이제 정의를 살펴봤으니, measurable function들이 어떠한 성질을 갖는지 살펴봅니다.

\thm{11.16} \(f\)가 measurable이면 \(\abs{f}\)도 measurable이다.

\pf 다음 관계로부터 자명하다.
\[
    \{x : \abs{f(x)} < a\} = \{x : f(x) < a\} \cap \{x : f(x) > -a\}.
\]

역은 성립할까요?

\rmk 역은 성립하지 않는다. Measurable하지 않은 \(S \subset (0, \infty)\) 위에서 함수 \(g\)를 다음과 같이 정의하자.
\[
    g(x) = \begin{cases}
        x & (x \in S) \\ -x & (x \notin S).
    \end{cases}
\]
그러면 모든 \(x \in \R\) 에 대해 \(\abs{g(x)} = x\) 이므로 \(\abs{g}\)는 measurable function이다. 하지만 \(\{x : g(x) > 0\} = \R \bs (-\infty, 0] = S\) 는 measurable이 아니므로 \(g\)는 measurable function이 아니다.

\prop. \(f, g\)가 measurable function이라 하자.
\begin{enumerate}
    \item \(\max\{f, g\}\), \(\min\{f, g\}\)는 measurable function이다.
    \item \(f^+ = \max\{f, 0\}\), \(f^- = -\min\{f, 0\}\) 는 measurable function이다.
\end{enumerate}

\pf 다음과 같이 적는다.
\[
    \begin{aligned}
        \{x : \max\{f, g\} > a\} & = \{x : f(x) > a\} \cup \{x : g(x) > a\} \\
        \{x : \min\{f, g\} < a\} & = \{x : f(x) < a\} \cup \{x : g(x) < a\}
    \end{aligned}
\]
그리고 (2)는 (1)에 의해 자명하다.

다음은 함수열의 경우입니다. Measurable 함수열의 극한함수도 measurable일까요?

\thm{11.17} \(\{f_n\}\)가 measurable 함수열이라 하자. 그러면
\[
    \sup_{n\in \N} f_n, \quad \inf_{n\in \N} f_n, \quad \limsup_{n \ra \infty} f_n, \quad \liminf_{n \ra \infty} f_n
\]
은 모두 measurable이다.

\pf 다음이 성립한다.
\[
    \inf f_n = -\sup\paren{-f_n}, \quad \limsup f_n = \inf_n \sup_{k\geq n} f_k, \quad \liminf f_n = -\limsup\paren{-f_n}.
\]
따라서 위 명제는 \(\sup f_n\)에 대해서만 보이면 충분하다. 이제 \(\sup f_n\)이 measurable function인 것은
\[
    \{x : \sup_{n\in\N} f_n(x) > a\} =  \bigcup_{n=1}^{\infty} \{x : f_n(x) > a\} \in \scr{F}
\]
로부터 당연하다.

\(\lim f_n\)이 존재하는 경우, 위 명제를 이용하면 \(\lim f_n = \limsup f_n = \liminf f_n\) 이기 때문에 다음을 알 수 있습니다. Measurability는 극한에 의해서 보존됩니다!

\cor 수렴하는 measurable 함수열의 극한함수는 measurable이다.

이제 마지막으로 measurable 함수의 합과 곱 또한 measurable이면 좋겠습니다. 각각 증명하는 것도 방법이지만, 두 경우를 한꺼번에 증명할 수 있는 방법이 있습니다.

\thm{11.18} \(X\)에서 정의된 실함수 \(f, g\)가 measurable이라 하자. 연속함수 \(F: \R^2 \ra \R\) 에 대하여 \(h(x) = F\big(f(x), g(x)\big)\) 는 measurable이다. 이로부터 \(f + g\)와 \(fg\)가 measurable임을 알 수 있다.\footnote{참고로 \(\infty - \infty\) 의 경우는 정의되지 않으므로 생각하지 않습니다.}

\pf \(a \in \R\) 에 대하여 \(G_a = \{(u, v)\in \R^2 : F(u, v) > a\}\) 로 정의합니다. 그러면 \(F\)가 연속이므로 \(G_a\)는 열린집합이고, \(G_a\) 열린구간의 합집합으로 적을 수 있다. 따라서 \(a_n, b_n, c_n, d_n\in \R\) 에 대하여
\[
    G_a = \ds \bigcup_{n=1}^{\infty} (a_n, b_n) \times (c_n, d_n)
\]
로 두면
\[
    \begin{aligned}
        \{x \in X : F\bigl(f(x), g(x)\bigr) > a\} = & \{x \in X : \bigl(f(x), g(x)\bigr) \in G_a\}                                              \\
        =                                           & \bigcup_{n=1}^{\infty} \{x \in X : a_n < f(x) < b_n,\, c_n < g(x) < d_n\}                 \\
        =                                           & \bigcup_{n=1}^{\infty} \{x \in X : a_n < f(x) < b_n\} \cap \{x \in X : c_n < g(x) < d_n\}
    \end{aligned}
\]
이다. 여기서 \(f, g\)가 measurable이므로 \(\{x \in X : F\bigl(f(x), g(x)\bigr) > a\}\)도 measurable이다. 이로부터 \(F(x, y) = x + y\), \(F(x, y) = xy\) 인 경우를 고려하면 \(f+g\), \(fg\)가 measurable임을 알 수 있다.

\dots

아래 내용은 Lebesgue integral의 정의에서 사용할 매우 중요한 building block입니다.

\defn. \note{Characteristic Function} 집합 \(E \subset X\) 의 \textbf{characteristic function} \(\chi_E\)는 다음과 같이 정의한다.
\[
    \chi_E(x) = \begin{cases}
        1 & (x\in E) \\ 0 & (x \notin E).
    \end{cases}
\]

참고로 characteristic function은 indicator function 등으로도 불리며, \(\mathbf{1}_E, K_E\)로 표기하는 경우도 있습니다.

\defn. \note{Simple Function} 함수 \(s: X\ra\R\) 의 치역이 유한집합이면 \textbf{simple function}이라 한다.

치역이 유한집합임을 이용하면 simple function은 다음과 같이 적을 수 있습니다.

\rmk 치역의 원소를 잡아 \(s(X) = \{c_1, c_2, \dots, c_n\}\) 로 두자. 여기서 \(E_i = s\inv(c_i)\) 로 두면 다음과 같이 적을 수 있다.
\[
    s(x) = \sum_{i=1}^{n} c_i \chi_{E_i}(x).
\]

이로부터 모든 simple function은 characteristic function의 linear combination으로 표현됨을 알 수 있습니다. 물론 \(E_i\)는 쌍마다 서로소입니다.

여기서 \(E_i\)에 measurable 조건이 추가되면, 정의에 의해 \(\chi_{E_i}\)도 measurable function입니다. 따라서 모든 measurable simple function을 measurable \(\chi_{E_i}\)의 linear combination으로 표현할 수 있습니다.

\dots

아래 정리는 simple function이 Lebesgue integral의 building block이 되는 이유를 잘 드러냅니다. 모든 함수는 simple function으로 근사할 수 있습니다.

\thm{11.20} \(f : X \ra \overline{\R}\) 라 두자. 모든 \(x \in X\) 에 대하여
\[
    \lim_{n \ra \infty} s_n(x) = f(x), \quad \abs{s_n(x)} \leq \abs{f(x)}
\]
인 simple 함수열 \(s_n\)이 존재한다. 여기서 추가로
\begin{enumerate}
    \item \(f\)가 유계이면 \(s_n\)은 \(f\)로 고르게 수렴한다.
    \item \(f\geq 0\) 이면 단조증가하는 함수열 \(s_n\)이 존재하며 \(\ds \sup_{n\in \N} s_n = f\) 이다.
    \item \textbf{\(f\)가 measurable이면 measurable simple 함수열 \(s_n\)이 존재한다.}
\end{enumerate}

\pf 우선 \(f \geq 0\) 인 경우부터 보인다. \(n \in \N\) 에 대하여 집합 \(E_{n, i}\)를 다음과 같이 정의한다.
\[
    E_{n, i} = \begin{cases}
        \left\{x : \dfrac{i}{2^n} \leq f(x) < \dfrac{i+1}{2^n}\right\} & (i = 0, 1, \dots, n\cdot 2^n - 1) \\
        \{x : f(x) \geq n\}                                            & (i = n\cdot 2^n)
    \end{cases}
\]
이를 이용하여
\[
    s_n(x) = \sum_{n=0}^{n\cdot 2^n} \frac{i}{2^n} \chi_{E_{n, i}} (x)
\]
로 두면 \(s_n\)은 simple function이다. 여기서 \(E_{n, i}\)와 \(s_n\)의 정의로부터 \(s_n(x) \leq f(x)\) 은 자연스럽게 얻어지고, \(x \in \{x : f(x) < n\}\) 에 대하여 \(\abs{f(x) - s_n(x)} \leq 2^{-n}\) 인 것도 알 수 있다. 여기서 \(f(x) \ra \infty\) 로 발산하는 부분이 존재하더라도, 충분히 큰 \(n\)에 대하여 \(\{x : f(x) \geq n\}\) 위에서는 \(s_n(x) = n \ra \infty\) 이므로 문제가 되지 않는다. 따라서
\[
    \lim_{n \ra \infty} s_n(x) = f(x), \quad (x \in X)
\]
라 할 수 있다.

(1)을 증명하기 위해 \(f\)가 유계임을 가정하면, 적당한 \(M > 0\) 에 대해 \(f(x) < M\) 이다. 그러면 충분히 큰 \(n\)에 대하여 \(\{x : f(x) < n\} = X\) 이므로 모든 \(x \in X\) 에 대해
\[
    \abs{f(x) - s_n(x)} \leq 2^{-n}
\]
가 되어 \(s_n\)이 \(f\)로 고르게 수렴함을 알 수 있다.

(2)의 경우 \(s_n\)의 정의에 의해 단조증가함을 알 수 있다. 여기서 \(f \geq 0\) 조건은 분명히 필요하다. \(s_n(x) \leq s_{n+1}(x)\) 이므로 당연히 \(\ds \sup_{n\in \N} s_n = f\) 이다.

(3)을 증명하기 위해 \(f\)가 measurable임을 가정하면 \(E_{n, i}\)도 measurable이므로 \(s_n\)은 measurable simple 함수열이 된다.

이제 일반적인 \(f\)에 대해서는 \(f = f^+ - f^-\) 로 적는다.\footnote{이 정의에서 \(\infty - \infty\) 가 나타나지 않음에 유의해야 합니다.} 그러면 앞서 증명한 사실을 이용해 \(g_n \ra f^+\), \(h_n \ra f^-\) 인 simple function \(g_n, h_n\)을 잡을 수 있다. 이제 \(s_n = g_n - h_n\) 으로 두면 \(\abs{s_n(x)} \leq \abs{f(x)}\) 가 성립하고, \(s_n \ra f\) 도 성립한다.

한편 이 정리를 이용하면 \(f + g\), \(fg\)가 measurable임을 증명하기 쉬워집니다. 단, \(f+g\), \(fg\)가 잘 정의되어야 합니다. 이는 \(\infty - \infty\) 와 같은 상황이 발생하지 않는 경우를 말합니다.

\cor \(f, g\)가 measurable이고 \(f + g\), \(fg\)가 잘 정의된다면, \(f+g\)와 \(fg\)는 measurable이다.

\pf \(f, g\)를 각각 measurable simple function \(f_n, g_n\)으로 근사한다. 그러면
\[
    f_n + g_n \ra f + g, \quad f_ng_n \ra fg
\]
이고 measurability는 극한에 의해 보존되므로 \(f+g, fg\)는 measurable이다.

\pagebreak
