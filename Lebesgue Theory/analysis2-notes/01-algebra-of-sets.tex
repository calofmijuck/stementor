\section*{Introduction}

이 시리즈에서는 르벡 적분을 다룹니다. 르벡 적분 또한 함수의 그래프와 \(x\)축 사이의 `부호 있는 넓이'를 측정한다는 점에서 리만 적분과 유사합니다. 하지만 리만 적분에서는 \(x\)축을 잘게 잘라 넓이를 근사했기 때문에 적분 가능성이 함수의 연속성에 크게 의존하게 됩니다. 르벡 적분에서는 \(y\)축을 잘게 자름으로써 이러한 문제를 해결하고, 적분의 수렴정리와 같은 유용한 결과를 쉽게 얻을 수 있습니다.

참고사항
\begin{itemize}
    \item 서울대학교 수리과학부 해석개론 및 연습 2 강의를 들으며 제가 정리한 강의 노트를 재구성했습니다. 강의 교재가 \textit{Principles of Mathematical Analysis} (Walter Rudin)이기 때문에 이 책을 많이 참고하였습니다.
    \item 수학 용어 특성상 번역이 마땅하지 않아 영어가 섞여있더라도 양해 부탁드립니다.
    \item 독자는 해석학에 대한 기초적인 지식을 갖고 있다고 가정합니다.
\end{itemize}

\dots

르벡 적분 이론에서는 확장실수체 \(\overline{\R}\)을 사용합니다.

\defn. \note{\(\overline{\R}\)} \textbf{확장실수체}(extended real numbers)는 다음과 같이 정의한다.
\[
    \overline{\R} = \R \cup \{-\infty, \infty\}.
\]
또한, 편의상 \(0\cdot \infty = 0\)으로 정의하고, \(\infty - \infty\)는 정의하지 않는다.

\(\infty + 1 = \infty\) 등과 같이 이외의 연산에 대해서는 자연스럽게 정의합니다. 다만 이제 \(\overline{\R}\)을 다루기 때문에 양변에서 항을 cancel 할 때 주의해야 합니다.

\pagebreak

\begin{center}
    \def\setA{(-0.866, 0.5) circle (\setradius cm)}
    \def\setB{(0.866, 0.5) circle (\setradius cm)}
    \def\setC{(0, -1) circle (\setradius cm)}
    \pgfmathsetmacro{\setradius}{1.5}

    \begin{tikzpicture}[scale=1.3]
        \begin{scope}
            \fill[color=gray!20, even odd rule] \setA \setB;
        \end{scope}

        \draw \setA \setB \setC;
        \node at (-1.366, 1) {\(A\)};
        \node at (1.366, 1) {\(B\)};
        \node at (0, -1.5) {\(C\)};
    \end{tikzpicture}
\end{center}

\section*{Algebra of Sets and Set Functions}

르벡 적분을 공부하기 위해서는 먼저 집합의 `길이' 개념을 공부해야 합니다. 그리고 집합의 `길이' 개념을 확립하기 위해서는 집합 간의 연산과 이에 대한 구조가 필요합니다.

\subsection*{Algebra of Sets}

\(\mc{R}\)이 집합의 모임이라고 하겠습니다. \(\mc{R} \neq \varnothing\) 임을 가정합니다.

\defn. \note{Ring} \(\mc{R}\)가 다음을 만족하면 \(\mc{R}\)를 \textbf{ring}이라고 한다.
\begin{center}
    모든 \(A, B \in \mc{R}\) 에 대하여 \(A \cup B \in \mc{R}\) 이고 \(A \bs B \in \mc{R}\) 이다.
\end{center}

합집합과 차집합에 닫혀 있으면 ring이 된다고 합니다. 다음을 관찰할 수 있습니다.

\prop. \(\mc{R}\)가 ring이라고 하자.
\begin{itemize}
    \item \(\mc{R}\neq \varnothing\) 이므로, \(A \in \mc{R}\) 를 잡을 수 있다. 따라서 \(A \bs A = \varnothing \in \mc{R}\) 이다.
    \item \(A, B \in \mc{R}\) 에 대하여 \(A \cap B = A \bs (A \bs B) \in \mc{R}\) 이므로 ring은 교집합에 대해서도 닫혀 있다.
\end{itemize}

합집합, 교집합, 차집합 등의 연산을 하다보면 자연스럽게 그 결과는 전체집합의 부분집합이 됩니다. 이를 모아 power set으로 정의하게 됩니다.

\defn. \note{Power Set} 집합 \(X\)에 대하여 power set \(\mc{P}(X)\)는 다음과 같이 정의한다.
\[
    \mc{P}(X) = \{A : A \subset X\}.
\]

Ring과 유사하지만 살짝 더 좋은 성질을 가진 구조를 가지고 논의를 전개합니다.

\defn. \note{Algebra} 다음 조건을 만족하는 \(\mc{F} \subset \mc{P}(X)\) 를 \textbf{algebra} on \(X\) 라고 한다.
\begin{enumerate}
    \item \(X \in \mc{F}\) 이다.
    \item \(A \in \mc{F}\) 이면 \(X \bs A \in \mc{F}\) 이다.
    \item \(A, B \in \mc{F}\) 이면 \(A\cup B \in \mc{F}\) 이다.
\end{enumerate}

\rmk 여집합의 경우 \(A^C = X \bs A\) 와 같이 표기합니다. 다음이 성립함을 이미 알고 있습니다.
\begin{center}
    \(A \cap B = (A^C \cup B^C)^C\),\quad \(A\bs B = A \cap B^C\).
\end{center}

그러므로 \(A, B \in \mc{F}\) 이면 \(A \cap B, A \bs B \in \mc{F}\) 입니다. 따라서 algebra의 경우 합집합, 교집합, 차집합, 여집합에 대해 모두 닫혀 있음을 알 수 있습니다. Ring 보다는 조금 더 다루기 편합니다.

자세히 살펴보니 ring의 정의와 유사한데, 1번 조건이 추가되었고 차집합이 \(X\)에 대한 차집합으로 바뀐 것을 확인할 수 있습니다. 실제로 다음이 성립하는 것을 확인할 수 있습니다.

\prop.
\begin{enumerate}
    \item \(\mc{R}\)이 algebra on \(X\)이면 \(\mc{R}\)은 ring이다.
    \item \(\mc{R} \subset \mc{P}(X)\)이 ring이고 \(X \in \mc{R}\) 이면 \(\mc{R}\)은 algebra on \(X\) 이다.
\end{enumerate}

조금만 더 확장해서 countable한 연산에 대해서도 허용하고 싶습니다.

\defn. \note{\(\sigma\)-ring} \(\mc{R}\)이 ring일 때, \(A_n \in \mc{R}\) (\(n = 1, 2, \dots\)) 에 대하여 \(\ds \bigcup_{n=1}^\infty A_n \in \mc{R}\) 이 성립하면 \(\mc{R}\)을 \textbf{\(\sigma\)-ring}이라 한다.

Countable한 합집합을 해도 닫혀 있다는 뜻입니다. 조금 생각해보면 마찬가지로 교집합에 대해서도 성립함을 알 수 있습니다.

\rmk 다음 성질
\[
    \bigcap_{n=1}^\infty A_n = A_1 \bs \bigcup_{n=1}^\infty (A_1 \bs A_n)
\]
을 이용하면 \(\mc{R}\)이 \(\sigma\)-ring이고 \(A_n \in \mc{R}\) 일 때 \(\ds \bigcap_{n=1}^\infty A_n \in \mc{R}\) 임을 알 수 있다.

마찬가지로 algebra도 정의할 수 있습니다.

\defn. \note{\(\sigma\)-algebra} \(\mc{F}\)가 algebra on \(X\)일 때, \(A_n \in \mc{F}\) (\(n = 1, 2, \dots\)) 에 대하여 \(\ds \bigcup_{n=1}^\infty A_n \in \mc{F}\) 가 성립하면 \(\mc{F}\)를 \textbf{\(\sigma\)-algebra}라 한다.

\(\sigma\)-algebra는 당연히 \(\sigma\)-ring이기 때문에 countable한 교집합을 해도 닫혀 있습니다.

\dots

집합 간의 연산을 정의했으니, 이제 집합의 `길이'를 정의할 준비가 되었습니다. 집합에 `길이'를 대응시키는 것은 곧 집합 위에서 함수를 정의하는 것과 같습니다. 이와 같은 맥락에서 set function 개념이 등장합니다.

\defn. \note{Set Function} \(\mc{R}\)이 ring on \(X\)라고 하자. 함수 \(\phi : \mc{R} \ra \overline{\R}\) 를 \(\mc{R}\) 위의 \textbf{set function}이라 한다.

정의역이 \(\mc{R}\)으로, 집합의 모임입니다. 즉 \(\phi\)는 집합을 받아 \(\overline{\R}\)과 대응시키는 함수임을 알 수 있습니다.

우리는 `길이' 함수를 정의하고자 합니다. `길이'는 보통 양수이기 때문에, \(\phi\)의 치역에 \(-\infty\)와 \(\infty\)가 동시에 포함되어 있는 경우는 제외합니다. 또한 \(\phi\)의 치역이 \(\{\infty\}\)이거나 \(\{-\infty\}\)인 경우도 생각하지 않습니다.

따라서, \(\phi(A) \in \R\) 인 \(A \in \mc{R}\)이 존재한다고 가정할 수 있습니다. 이 사실은 양변에서 \(\phi(A)\)를 cancel 할 때 사용됩니다.

\defn. \(\phi\)는 \(\mc{R}\) 위의 set function이다.
\begin{enumerate}
    \item 서로소인 두 집합 \(A, B \in \mc{R}\) 에 대하여
          \[
              \phi(A\cup B) = \phi(A) + \phi(B)
          \]
          이면 \(\phi\)는 \textbf{additive}하다.

    \item 쌍마다 서로소인 집합 \(A_i \in \mc{R}\) 에 대하여
          \[
              \phi\paren{\bigcup_{i=1}^\infty A_i} = \sum_{i=1}^\infty \phi(A_i)
          \]
          이고 \(\ds \bigcup_{i=1}^\infty A_i \in \mc{R}\) 이면\footnote{\(\sigma\)-ring 이면 불필요한 조건이지만, 일반적인 ring에 대해서는 필요한 조건입니다.} \(\phi\)는 \textbf{countably additive} (\(\sigma\)-additive) 하다.
\end{enumerate}

이제 `길이'의 개념을 나타내는 함수를 정의합니다. 이 함수는 측도(measure)라고 합니다.

\defn. \note{Measure} \(\sigma\)-ring \(\mc{R}\)에 대하여, \(\mc{R}\) 위의 set function \(\mu\)가 countably additive이고 치역이 \([0, \infty]\)이면 \(\mu\)를 measure on \(\mc{R}\)이라 한다.

치역이 음이 아닌 실수와 무한대인 것은 `길이'의 개념을 나타내기 위해서입니다. 또한 countable 성질을 가져가고 싶은 이유를 이제 설명할 수 있습니다. 열린집합 \(U \subset \R\) 은 서로소인 열린 구간의 countable한 합집합으로 표현할 수 있습니다.\footnote{증명은 해석개론, 김김계 책을 참고해주세요. 각 구간마다 유리수를 택할 수 있고, 유리수는 countable이기 때문에...} 실수의 열린 부분집합에 대해 먼저 `길이'를 정의하고 이를 다른 부분집합으로 확장하는 논리 전개 방식을 택할 예정이기 때문에, countable 조건이 필요한 것입니다.

\medskip

\rmk
\begin{enumerate}
    \item \(\phi\)가 additive이면 쌍마다 서로소인 \(A_i \in \mc{R}\) 에 대하여 다음이 성립한다.
          \[
              \phi\paren{\bigcup_{i=1}^n A_i} = \sum_{i=1}^n \phi(A_i).
          \]
          이 성질을 \textit{finite additivity}라 부르고, \(\phi\)는 \textit{finitely additive}하다고 한다.

    \item \(\phi(\varnothing) = 0\) 이다. \(\phi(A) = \phi(A \cup \varnothing)\) 라고 적고, 양변에서 \(\phi(A) \in \R\) 를 지울 수 있다.
\end{enumerate}

\(\phi(A) \in \R\) 인 \(A \in \mc{R}\)이 존재한다는 가정을 사용했습니다. 앞서 언급한 것처럼, extended real number가 나오기 때문에 뺄셈에 조심해야 합니다.

\medskip

\defn. \(\mu\)가 measure on \(\sigma\)-algebra \(\mc{F} \subset \mc{P}(X)\) 라 하자.
\begin{enumerate}
    \item \(\mu\)가 \textbf{finite} 하다. \miff 모든 \(X \in \mc{F}\) 에 대하여 \(\mu(X) < \infty\) 이다.
    \item \(\mu\)가 \textbf{\(\sigma\)-finite} 하다. \miff 집합열 \(F_1 \subset F_2 \subset \cdots\) 가 존재하여 \(\mu(F_i) < \infty \) 이고 \(\ds \bigcup_{i=1}^\infty F_i = X\) 이다.
\end{enumerate}

\subsection*{Some Basic Properties of Set Functions}

\(\phi\)가 set function이라 하자.

\begin{itemize}
    \item \(\phi\)가 ring \(\mc{R}\) 위에서 countably additive이면 \(\phi\)는 additive이다.

    \item \(\phi\)가 ring \(\mc{R}\) 위에서 additive이면, \(A, B \in \mc{R}\) 에 대하여
          \[
              \phi(A\cup B) + \phi(A\cap B) = \phi(A) + \phi(B)
          \]
          가 성립한다.\footnote{확률의 덧셈정리와 유사합니다. 확률론 또한 measure theory와 관련이 깊습니다.}

    \item \(\phi\)가 ring \(\mc{R}\) 위에서 additive이면, \(A_1 \subset A_2\) 인 \(A_1, A_2 \in \mc{R}\) 에 대하여
          \[
              \phi(A_2) = \phi(A_2 \bs A_1) + \phi(A_1)
          \]
          가 성립한다. 따라서,
          \begin{enumerate}
              \item \(\phi \geq 0\) 이면 \(\phi(A_1) \leq \phi(A_2)\) 이다. (단조성)
              \item \(\abs{\phi(A_1)} < \infty\) 이면 \(\phi(A_2 \bs A_1) = \phi(A_2) - \phi(A_1)\) 이다.\footnote{무한하지 않다는 조건이 있어야 이항이 가능합니다.}
          \end{enumerate}

    \item \(\phi\)가 additive이고 \(\phi \geq 0\) 이면 \(A, B \in \mc{R}\) 에 대하여
          \[
              \phi(A\cup B) \leq \phi(A) + \phi(B)
          \]
          가 성립한다. 귀납법을 적용하면, 모든 \(A_i \in \mc{R}\)에 대하여
          \[
              \phi\paren{\bigcup_{n=1}^m A_n} \leq \sum_{n=1}^m \phi(A_n)
          \]
          가 성립한다. 이 때 \(A_i\)가 반드시 쌍마다 서로소일 필요는 없다. 이 성질을 \textit{finite subadditivity}라 한다.
\end{itemize}

마지막으로 measure와 관련된 정리를 소개합니다.

\thm. \(\mu\)가 \(\sigma\)-algebra \(\mc{F}\)의 measure라 하자. \(A_n \in \mc{F}\) 에 대하여 \(A_1 \subset A_2 \subset \cdots\) 이면
\[
    \lim_{n\ra\infty} \mu(A_n) = \mu\paren{\bigcup_{n=1}^\infty A_n}
\]
이 성립한다.

\pf \(B_1 = A_1\), \(n \geq 2\) 에 대해 \(B_n = A_n \bs A_{n-1}\) 로 두자. \(B_n\)은 쌍마다 서로소임이 자명하다. 따라서,
\[
    \mu(A_n) = \mu\paren{\bigcup_{k=1}^n B_k} = \sum_{k=1}^n \mu(B_k)
\]
이고, measure의 countable additivity를 이용하여
\[
    \lim_{n\ra\infty} \mu(A_n) = \lim_{n\ra\infty} \sum_{k=1}^n \mu(B_k) = \sum_{n=1}^\infty \mu(B_n) = \mu\paren{\bigcup_{n=1}^{\infty} B_n} = \mu\paren{\bigcup_{n=1}^\infty A_n}
\]
임을 알 수 있다. 마지막 등호에서는 \(\ds \bigcup_{n=1}^\infty A_n = \bigcup_{n=1}^\infty B_n\) 임을 이용한다.

왠지 위 조건을 뒤집어서 \(A_1 \supseteq A_2 \supseteq \cdots\) 인 경우 교집합에 대해서도 성립하면 좋을 것 같습니다.
\[
    \lim_{n\ra\infty} \mu(A_n) = \mu\paren{\bigcap_{n=1}^\infty A_n}.
\]
하지만 안타깝게도 조건이 부족합니다. \(\mu(A_1) < \infty\) 라는 추가 조건이 필요합니다. 반례는 \(A_n = [n, \infty)\)를 생각해보면 됩니다. 정리의 정확한 서술은 다음과 같습니다. 증명은 연습문제로 남깁니다.

\thm. \(\mu\)가 \(\sigma\)-algebra \(\mc{F}\)의 measure라 하자. \(A_n \in \mc{F}\) 에 대하여 \(A_1 \supseteq A_2 \supseteq \cdots\) 이고 \(\mu(A_1) < \infty\) 이면
\[
    \lim_{n\ra\infty} \mu(A_n) = \mu\paren{\bigcap_{n=1}^\infty A_n}
\]
이 성립한다.

이 두 정리를 continuity of measure라고 합니다. 함수가 연속이면 극한이 함수 안으로 들어갈 수 있는 성질과 유사하여 이와 같은 이름이 붙었습니다. 어떤 책에서는 \(A_1 \subset A_2 \subset \cdots\) 조건을 \(A_n \nearrow \bigcup_n A_n\) 라 표현하기도 합니다. 그래서 이 조건에 대한 정리를 continuity from below라 하기도 합니다. 마찬가지로 \(A_1 \supseteq A_2 \supseteq \cdots\) 조건을 \(A_n \searrow \bigcap_n A_n\) 로 적고 이에 대한 정리를 continuity from above라 합니다.

\dots

이제 measure의 개념을 정리했으니 다음 글에서는 본격적으로 집합을 재보려고 합니다. 우리의 목표는 \(\R^p\)에서 measure를 정의하는 것입니다. 우선 쉽게 잴 수 있는 집합들부터 고려할 것입니다.

\pagebreak
