\section*{November 1st, 2022}

\((X, \scr{F}, \mu)\) 라고 계속 가정합니다. \(\scr{F}\)는 \(\sigma\)-algebra on \(X\), \(\mu\)는 measure on \(\scr{F}\) 입니다.

If \(E \in \scr{F}\), we can consider (\(X, \scr{F}_E, \mu_E\)) where
\[
    \scr{F}_E = \{A \cap E : A \in \scr{F}\}, \quad \mu_E = \mu|_{\scr{F}_E},
\]
and develop the integration theory with \(\int = \int_E\).

Instead, we develop integration theory with \(\int = \int_X\) and set \(\ds \int_E f \d{\mu} = \int f \chi _E \d{\mu}\).

\medskip

\note{Step 1} For \(A \in \scr{F}\), we define \(\ds \int \chi_A \d{\mu} = \mu(A)\).

\note{Step 2} Let \(f: X \ra [0, \infty)\) be a measurable simple function. \\
Then there exists pairwise disjoint \(\seq{A_k}_{k=1}^n\) in \(\scr{F}\) and \(\seq{a_k}_{k=1}^n\) in \([0, \infty)\)\footnote{책에서는 \((0, \infty)\)이긴 한데, 우리는 \(0 \cdot\infty = 0\)이라 생각하고 \(\bigcup_{k=1}^n A_k = X\) 라고 생각할게요.} such that
\[
    f(x) = \sum_{k=1}^n a_k \chi_{A_k}.
\]
Then we can define
\[ \tag{\mast}
    \int f\d{\mu} = \sum_{k=1}^n a_k \mu(A_k) \in [0, \infty].
\]

이런 정의를 보면 여러분이 제일 먼저 생각해야 하는 것이 well-definedness 입니다!

\rmk (\mast) is well-defined for all measurable simple functions.

\pf Let
\[
    f(x) = \sum_{k=1}^n a_k \chi_{A_k} = \sum_{i=1}^m b_i \chi_{B_i},
\]
where \(0\leq a_k, b_i < \infty\) and \(A_k, B_i \in \scr{F}\) are partitions of \(X\).\footnote{Pairwise disjoint and their union is \(X\).} Let \(C_{k, i} = A_k \cap B_i\). Then
\[
    \sum_{k=1}^n a_k \mu(A_k) = \sum_{k=1}^n a_k \mu\paren{A_k \cap \bigcup_{i=1}^m B_i} = \sum_{k=1}^n \sum_{i=1}^m a_k \mu(C_{k, i}),
\]
\[
    \sum_{i=1}^m b_i \mu(B_i) = \sum_{i=1}^{m} b_i \mu\paren{B_i \cap \bigcup_{k=1}^n A_k}= \sum_{i=1}^m \sum_{k=1}^n b_i \mu(C_{k, i}).
\]
If \(C_{k, i} \neq \varnothing\), then \(a_k = b_i\).\footnote{For \(x \in C_{k, i}\), \(f(x) = a_k = b_i\).} If \(C_{k, i} = \varnothing\), then \(\mu(C_{k, i}) = 0\). Therefore
\[
    b_i \mu(C_{k, i}) = a_k \mu(C_{k, i}), \quad (\forall k, i) \implies \int f \d\mu = \sum_{k=1}^n a_k \mu(A_k) = \sum_{i=1}^m b_i \mu(B_i).
\]

\pagebreak

적분은 선형이고, monotonicity를 항상 유지합니다.

\rmk For \(a, b \in [0, \infty)\) and measurable simple functions \(f, g \geq 0\),
\[
    \int \paren{af + bg} \d{\mu} = a \int f \d{\mu} + b \int g \d{\mu}.
\]

\pf Let
\[
    f = \sum_{j=1}^m y_j \chi_{A_j}, \quad g = \sum_{k=1}^n z_k \chi_{B_k}
\]
where \(A_j, B_k\) is a partition of \(X\) and \(y_j, z_k \geq 0\). Let \(C_{j, k} = A_j \cap B_k\).
Then
\[
    \begin{aligned}
        a \int f \d{\mu} + b \int g \d{\mu} & = \sum_{j} ay_j \mu(A_j) + \sum_k b z_k \mu(B_k)                                 \\
                                            & = \sum_{j} ay_j \sum_k \mu(A_j \cap B_k) + \sum_k b z_k \sum_j \mu(B_k \cap A_j) \\
                                            & = \sum_{j} \sum_k ay_j \mu(C_{j, k}) + \sum_k \sum_j b z_k \mu(C_{j, k})         \\
                                            & = \sum_{j, k} (ay_j + bz_k) \mu(C_{j, k}) = \int \paren{af + bg} \d{\mu}.
    \end{aligned}
\]

\rmk If \(f \geq g \geq 0\) are measurable simple functions, \(\ds \int f \d{\mu} \geq \int g \d{\mu}\).

\pf Check from definition, or check that \(f - g \geq 0\) is simple and measurable.
\[
    \int f \d{\mu} = \int \left[g + (f - g)\right] \d{\mu} = \int g\d{\mu} + \int (f - g) \d{\mu} \geq \int g \d{\mu} \geq 0.\footnote{적분값이 무한대일 수 있으니 함부로 이항하면 안됩니다.}
\]

\note{Step 3} Let \(f: X \ra [0, \infty]\) be a measurable function. Define
\[
    \int f \d{\mu} = \sup\left\{\int h \d{\mu}: 0\leq h \leq f, h \text{ is measurable and simple}\right\}.
\]
Note that if \(f\) is simple, this accords with the definition in {\sffamily Step 2}.

For measurable \(f \geq g \geq 0\),
\[
    \int g \d{\mu} = \sup_{0\leq h\leq g} \int h\d{\mu} \leq \sup_{0 \leq h\leq f} \int h \d{\mu} = \int f \d{\mu}.
\]

\pagebreak

이걸 먼저 증명하면 유용해서 잠깐 이걸 먼저 할게요.

\thm{11.28} \note{Monotone Convergence Theorem} Let \(f_n: X \ra [0, \infty]\) be measurable functions\footnote{Non-negative인 것이 중요합니다!} and \(f_n(x) \leq f_{n+1}(x)\) for all \(x \in X\). Let \[
    \lim_{n\ra\infty} f_n(x) = \sup_{n} f_n(x) = f(x).
\]
Then,
\[
    \int f \d{\mu} = \lim_{n\ra\infty} \int f_n \d{\mu} = \sup_{n} \int f_n \d{\mu}.
\]

\pf \\
\note{\(\geq\)} Since \(f_n(x) \leq f(x)\), \(\int f_n \d{\mu} \leq \int f \d{\mu}\) for all \(n\in \N\) by monotonicity. Therefore
\[
    \sup_n \int f_n \d{\mu} \leq \int f \d{\mu}.
\]

\note{\(\leq\)} Let \(c \in (0, 1)\) (we will let \(c \nearrow 1\)). Let \(0 \leq s \leq f\) where \(s\) is simple and measurable. We have \(c \cdot s(x) < f(x)\) for all \(x \in X\). Let \(E_n = \{x \in X : f_n(x) \geq cs(x)\} \in \scr{F}\).\footnote{\(f_n(x) - cs(x)\) is a measurable function.} Since \(f_n\) is increasing, \(E_n\subset E_{n+1} \subset \cdots\), and since \(f_n \ra f\), \(\bigcup_{n=1}^\infty E_n = X\). For every \(x\), \(\exists N\) such that \(f(x) \geq f_n(x) > cs(x)\) for all \(n \geq N\). Since \(f_n \geq f_n \chi_{E_n} \geq cs \chi_{E_n}\),
\[ \tag{\mstar}
    \int f_n \d{\mu} \geq \int f_n \chi_{E_n} \d{\mu} \geq c\int s \chi_{E_n} \d{\mu},
\]
where \(s, \chi_{E_n}\) are simple. So we can write \(s = \sum_{k=0}^m y_k \chi_{A_k}\), then
\[
    s\chi_{E_n} = \sum_{k=0}^m y_k \chi_{A_k\cap E_n} \implies \int s \chi_{E_n} \d{\mu} = \sum_{k=0}^m y_k \mu(A_k\cap E_n).
\]
\(A_k\cap E_n \nearrow A_k\) as \(n\ra\infty\). By continuity of measure, \(\mu(A_k \cap E_n) \nearrow \mu(A_k)\) and
\[
    \lim_{n\ra\infty} \int s \chi_{E_n}\d{\mu} = \int s \d{\mu}.
\]
By (\mstar),
\[
    \lim_{n\ra\infty} \int f_n \d{\mu} \geq c\int s \d{\mu}.
\]
Let \(c \nearrow 1\) and take supremum over \(0\leq s\leq f\), then
\[
    \lim_{n\ra\infty} \int f_n \d{\mu} \geq \sup_{0\leq s\leq f} \int s \d{\mu} = \int f \d{\mu}.
\]

\pagebreak

\rmk If \(f \geq 0\) is measurable, we have already constructed measurable simple functions \(s_n\) such that \(s_n(x) \leq s_{n+1}(x)\). ({\sffamily Theorem 11.20}) Let \(f(x) = \ds \lim_{n\ra\infty} s_n(x)\). Observe that
\[
    \int_E s_n \d{\mu} = \sum_{i=1}^{n2^n} \frac{i - 1}{2^n}\mu\paren{\left\{x \in E : \frac{i-1}{2^n} \leq f(x) \leq \frac{i}{2^n}\right\}} + n\mu(\{x \in E : f(x)\geq n\}).
\]
Then by MCT,
\[
    \int_E f \d{\mu} = \lim_{n\ra\infty} \int_E s_n \d{\mu}.
\]

\bigskip

Lebesgue integral이 Riemann이랑 어떻게 다른가? Riemann은 domain을 자르고 upper/lower sum을 고려해서 만든다면, Lebesgue의 경우 range를 잘라서, 자른 range의 preimage에 대한 measure를 이용해 preimage의 길이를 재는 것입니다. Riemann과 마찬가지로 domain의 길이와 높이를 곱해 모두 더하는 idea는 동일합니다.

\bigskip

\defn. If \(E \in \mc{F}\), and \(f \geq 0\) is measurable, then
\[
    \int_E f \d{\mu} = \int f \chi_{E} \d{\mu}.
\]

\rmk If \(0 \leq f_n \nearrow f\) on \(E\), then \(0\leq f_n \chi_E \nearrow f \chi_E\). So MCT also holds on \(E\).
\begin{center}
    \(0\leq f_n \nearrow f\) on \(E\) \(\implies \ds \lim_{n\ra\infty} \int_E f_n \d{\mu} = \int_E f \d{\mu}\).
\end{center}

\rmk Counterexample of MCT: \(\chi_{[n, \infty)} \searrow 0\) as \(n \ra \infty\). For Lebesgue measure \(m\),
\[
    \infty = \int \chi_{[n, \infty)} \d{m} \neq \int 0 \d{m} = 0.
\]

\rmk Suppose \(f, g \geq 0\) be measurable functions, and \(\alpha, \beta \in [0, \infty)\). Then
\[
    \int_E \paren{\alpha f + \beta g} \d{\mu} = \alpha \int_E f \d{\mu} + \beta \int_E g\d{\mu}.
\]

\pf Take \(0 \leq f_n \nearrow f\) and \(0 \leq g_n\nearrow g\) where \(f_n, g_n\) are measurable simple functions. Then \(\alpha f_n + \beta g_n \nearrow \alpha f + \beta g\) and is simple. So by MCT,
\[
    \int_E \paren{\alpha f_n + \beta g_n} \d{\mu} = \alpha \int_E f_n \d{\mu} + \beta \int_E g_n \d{\mu} \ra \alpha \int_E f \d{\mu} + \beta \int_E g\d{\mu}.
\]

\thm{11.30} For measurable \(f_n: X \ra [0, \infty]\), \(\ds \sum_{n=1}^\infty f_n\) is measurable and by MCT,
\[
    \int_E \sum_{n=1}^\infty f_n \d{\mu} = \sum_{n=1}^\infty \int_E f_n \d{\mu}.
\]

\pagebreak
