\section*{Introduction}

이 시리즈에서는 르벡 적분을 다룹니다. 르벡 적분 또한 함수의 그래프와 \(x\)축 사이의 `부호 있는 넓이'를 측정한다는 점에서 리만 적분과 유사합니다. 하지만 리만 적분에서는 \(x\)축을 잘게 잘라 넓이를 근사했기 때문에 적분 가능성이 함수의 연속성에 크게 의존하게 됩니다. 르벡 적분에서는 \(y\)축을 잘게 자름으로써 이러한 문제를 해결하고, 적분의 수렴정리와 같은 유용한 결과를 쉽게 얻을 수 있습니다.

참고사항
\begin{itemize}
    \item 서울대학교 수리과학부 해석개론 및 연습 2 강의를 들으며 제가 정리한 강의 노트를 재구성했습니다. 강의 교재가 \textit{Principles of Mathematical Analysis} (Walter Rudin)이기 때문에 이 책을 많이 참고하였습니다.
    \item 수학 용어 특성상 번역이 마땅하지 않아 영어가 섞여있더라도 양해 부탁드립니다.
    \item 독자는 해석학에 대한 기초적인 지식을 갖고 있다고 가정합니다.
\end{itemize}

\dots

르벡 적분 이론에서는 확장실수체 \(\overline{\R}\)을 사용합니다.

\defn. \note{\(\overline{\R}\)} \textbf{확장실수체}(extended real numbers)는 다음과 같이 정의한다.
\[
    \overline{\R} = \R \cup \{-\infty, \infty\}.
\]
또한, 편의상 \(0\cdot \infty = 0\)으로 정의하고, \(\infty - \infty\)는 정의하지 않는다.

\(\infty + 1 = \infty\) 등과 같이 이외의 연산에 대해서는 자연스럽게 정의합니다. 다만 이제 \(\overline{\R}\)을 다루기 때문에 양변에서 항을 cancel 할 때 주의해야 합니다.

\pagebreak
